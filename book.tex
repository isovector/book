\documentclass[12pt]{book}

\usepackage[margin=1in]{geometry}
\usepackage{amssymb, amsmath, amsfonts}
\usepackage{titlecaps}
\usepackage{xargs}
\usepackage[colorinlistoftodos,prependcaption,textsize=tiny]{todonotes}

\title{Ideas Unthinkable}
\author{Sandy Maguire}
\date{\today}

\newcommand{\difficult}{}
\newcommandx{\unsure}[2][1=]{\todo[linecolor=red,backgroundcolor=red!25,bordercolor=red,#1]{#2}}
\newcommandx{\change}[2][1=]{\todo[linecolor=blue,backgroundcolor=blue!25,bordercolor=blue,#1]{#2}}
\newcommandx{\info}[2][1=]{\todo[linecolor=green,backgroundcolor=green!25,bordercolor=green,#1]{#2}}
\newcommandx{\improvement}[2][1=]{\todo[linecolor=purple,backgroundcolor=purple!25,bordercolor=purple,#1]{#2}}
\newcommandx{\thiswillnotshow}[2][1=]{\todo[disable,#1]{#2}}
\newcommand{\aside}[1]{#1}
\renewcommand{\value}[1]{\textit{#1}}
\newcommand{\type}[1]{\texttt{#1}}
\newcommand{\what}{}
\newcommand{\defn}[2]{\renewcommand{\what}{#1}\textbf{Definition "\titlecap{#1}":} #2}
\renewcommand{\cite}[1]{}



\begin{document} \maketitle

\chapter{Introduction}
Iktsuarpok is an Inuit word which roughly translates in English to "the frustration of waiting for someone to turn up."
Think about that feeling; it's something you're almost certainly familiar with, but, I'd suspect, never in the sense of
iktsuarpok. You probably categorized the feeling as "frustration" or "irritability", both of which are close, but
neither of which is as accurate as iktusarpok. English doesn't really have a word for this feeling, but somehow, we
manage to get by. Our existing words do a adequate-enough job of expressing our emotions, and so we don't see the need
to spend the effort finding and spreading a new word just for this purpose.

The Sapir-Whorf hypothesis\improvement{Be more science-y here}(or linguistic relativism), a theory widely considered
to be true by scholars of linguistics, states that the language you speak influences the thoughts you can have. For a
particularly illustrating example of this hypothesis, quickly think to yourself which direction is south-west relative
to you right this instant.  Unless you have recently looked at a map, your orienteering isn't likely to be very
accurate. Humans, it would seem, do not have very good internal compasses. However, this claim turns out to have very
few legs to stand on indeed; the Australian Aboriginal people the Guugu Yimithirr
\cite{http://en.wikipedia.org/wiki/Relative_direction} have no issue with this task whatsoever, and the reason why is
quite illuminating: they do not have words for "left" or "right". When asking a Guugu Yimithirr which hand he writes
with, he might respond with "my Eastern-most hand." The fact that their language lacks words for relative directions
means that the Guugu Yimithirr need an alternative means of describing direction, and so they instead keep track
mentally of the cardinal directions to provide the same purpose.

Isn't it quite remarkable to realize that, of all of the \unsure{is this number accurate?}millions of words in the
English language, there still exist ideas that are difficult to express? Ideas that would take so long to describe that
by the time you got to the end of your sentence, people would have forgotten the beginning? Linguistic relativism
implies that this indeed must be the case; that there are thoughts English is poorly adapted to handle.

\improvement{I don't like the way this is worded.}That's What this book is for; to teach you a language capable of
expressing the ideas you never even knew were inexpressible. As it turns out, these ideas are \textit{profoundly
powerful}; they are the reason that engineers and computer programmers are so highly paid. Unfortunately, as it is with
all things, we must learn to walk before we can run. It is an inescapable consequence of this that a lot of the ideas
will seem obvious at first, but that is only because you have had decades of practice with your current thinking
patterns.

I liken the experience of this as similar to learning to ride a bicycle\todo{maybe a car analogy would make more sense}.
When you first start, you're not any good at it, and every time you get on there is a non-zero chance that you will end
up hurting yourself. For all intents and purposes, early on, it's easier to get where you're going by walking than by
riding your bicycle. However, once you've gotten familiar with the bike, and have learned to not fall over, even in the
absence of your training wheels, you're more mobile than ever before. I think this is a fair metaphor for what we're
trying to do here. Give this book the benefit of the doubt; just because your current system works better at first,
doesn't necessarily it will be better in the long run.\improvement{this sounds patronizing}

\section{Things I want to say that I stole from that stats book}
We have tried to write this book in an intuitive fashion, emphasizing concepts rather than mathematical details.
The symbol \difficult indicates a technically difficult section, one that can be skipped without interrupting the flow
of the discussion.



\chapter{Spaces: The Language of Possibilities}
\chapter{Types: The Language of Structure}
\section{Motivation}
At its core, computation\info{and all of the things that go along with it} really nothing more than a series of
transformations from one thing to another, and from the result of that transformation to something further along. This
is pretty easy to observe in the real world; every firm involved in manufacturing chain is a good example. The mining
corporations transform the raw earth into crude materials, which are then taken by the chemical engineers and
transformed into refined materials. These refined materials are sold to factories, which transform them into assembled
goods, and will usually sell them to other factories to make other assembled goods. Each individual organization
provides one individual piece of the overall manufacturing transformation -- the whole of which is made up of nothing
but the individual pieces. The structure of a transformation is recursive: it is made up of smaller transformations, and
makes up larger ones.\improvement{Drive in the point that this goes further in both directions: eg. every person at an
organization is responsible to transform their pay into something valuable for the company.}

In fact, one might be tempted to view ones own life as a transformation, from what they were born with to what they want
to do with it. Though human-level problems are unfortunately all-too-often too complicated for our
mathematical\unsure{do we want to use this word?} tools, taking this view is likely apt to help people find their
way.\change{blegh. not sure this is thematically appropriate}

However, in order to study large transformations of any value whatsoever, we must first present ourselves with toy
problems and analyze how they work, trusting that our discoveries will be useful when dealing with more complicated
examples. It is hard to perform a transformation if you are uncertain about what it is you're working with. The study of
types is thus the necessary bookkeeping of \textit{what we're dealing with} and \textit{what we can do with it}. The
reader is encouraged to thoroughly understand (and ideally, have internalized) this chapter before proceeding, as the
remainder of the book builds heavily upon it.


\section{Types and Values}
The study of types is motivated as a necessary component of studying \textit{values}, and as such, it would seem to make
sense to first consider values.

\defn{value}{A \what is a particular noun, i.e. it's a thing that exists.}

\aside{Such a definition seems unnecessarily vague, but this is in fact intentional, and, as we will see, rather
characteristic of most of the definitions in this book. The concepts we are trying to learn words for are purposefully
vague; we are attempting to build mental scaffolding that works regardless of what in question we are thinking about.}

For the time being, we will consider values to be things that exist in the real world, things you can physically
experience. For example, my copy of \value{Crime and Punishment} is a value. So is \value{George W. Bush}, and the color
\value{brown}. The word \value{"Charlie"} is a value, and so is my friend \value{Charlie} -- but these two values are
distinct from one another, for reasons that we will see shortly.

$$
\text{\texttt{Beverage}} \to \verb"Number"
$$


\chapter{Algorithms: The Language of Decision}
\chapter{Monads: The Language of Progression}
\chapter{Isomorphisms: The Language of Identity}
\chapter{Asymptotes: The Language of Refinement}


\end{document}

